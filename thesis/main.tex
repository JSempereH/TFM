\documentclass[oneside, openany]{book}
\usepackage[utf8]{inputenc}
\usepackage[backend=bibtex]{biblatex}
\usepackage{amssymb, amsmath, mathtools}
\newtheorem{definition}{Definition}

\usepackage[left=1in, right=1in, top=1in, bottom=1in, includefoot, headheight=13.6pt]{geometry}
\usepackage[T1]{fontenc}

\usepackage{algorithm}
\usepackage[noend]{algpseudocode}
\renewcommand{\algorithmicrequire}{\textbf{Input:}}
\renewcommand{\algorithmicensure}{\textbf{Output:}}

\usepackage{graphicx}
\usepackage{subcaption}
\usepackage{tikz}

\usepackage{url}
\usepackage{pdfpages}

\usepackage{adjustbox}
\usepackage{booktabs}

\usepackage[11pt]{moresize}

\let\cleardoublepage=\clearpage

\addbibresource{TFM.bib}

\begin{document}

\includepdf[pages=-]{TECI_Caratula.pdf}
\small{
The aim of this paper is to introduce some topics related to Privacy Enhancing Machine Learning (or more generally, Privacy Enhancing Technologies), which I was introduced to during my internship at GMV while studying for my Master’s in Computational Statistical Information Processing. With this, I hope to simply explain some techniques and algorithms used in practice, as well as the challenges and issues that may arise when applying them.\\
In the first section, there will be a brief explanation of what Privacy Enhancing Machine Learning is, as well as its context. Additionally, some basic concepts like the multilayer perceptron or convolutional networks, which will be used later in the paper, will be briefly defined. Some optimizers that will be useful for understanding some algorithms explained in section 2 will also be outlined.\\
Section 2 will focus on defining and explaining the challenges of Federated Learning, explaining some of its most prominent algorithms, and showing experiments conducted to compare the performance of each one under the same configuration. In particular, I will focus on the effect of potential data asymmetries in a federated environment on the performance of each algorithm, thus studying statistical heterogeneity (one of the most relevant challenges in this branch of distributed learning).\\
In section 3, there will be a brief mention of what it means to train models on vertically partitioned data and the formulation of SplitNN. This section could be greatly expanded if Multi-Party Computation, which is one of the key technologies within the field of PETs, were introduced. However, this would require introducing several cryptographic concepts and would go beyond the statistical context of the master’s program.\\
In section 4, other somewhat more advanced concepts, such as Differential Privacy, are introduced, to which a couple of algorithms related to the training of Deep Learning models are dedicated, as well as the use of generative models. A technique for generating tabular data is introduced, and a possible use in the field of PETs is mentioned, albeit with certain criticisms regarding its usability and security.

Following the guidelines, the following sections are defined:

\textbf{Problem Statement:} In recent years, different technologies have emerged that seek to use current discriminative and generative models trained with confidential or private data (banking data, health records, etc.). This can enable collaboration between companies and universities to jointly improve models without the need to share data, as well as boost scientific research on rare diseases or data where each research center has few data points or regulations require greater confidentiality in their processing. Currently, there are many alternatives depending on the use case, and it is still being studied to what extent they are secure.

\textbf{Development:} The development of this paper will consist of the introduction of different techniques related to subjects studied in class, particularly the subjects of Neural Networks, Tools for Big Data, Optimization Techniques, Pattern Recognition Techniques, and Statistical Software. The experiments conducted to compare algorithms are my own work (unless otherwise stated), and both the code and results can be found in the GitHub repository: \url{github.com/JSempereH/TFM.}

\textbf{Conclusion:} Only a few algorithms and methods could be studied. The conclusions of the experiments related to Federated Learning in section 2 align with the literature: a different distribution of data among computing nodes greatly degrades the de facto algorithm FedAvg, with FedProx being a better alternative, though not the only one. Regarding the different techniques related to PETs, only some that I worked with during my internship and that are related to Computational Statistical Information Processing could be introduced.

There is currently a great economic interest in preserving privacy (due to penalties, especially in the European territory). The technologies presented here are promising but not magical. They require research, careful implementation, and supervision. They are not free from criticism and possible security flaws in certain cases, in addition to a degradation of models compared to centralized training. The aim is not to present these topics as the solution to privacy, but rather to understand, even if superficially, what they are about and what they seek to solve. I hope I have achieved this, or at least provided a good bibliography to consult material on each topic.\\
This work has been made possible thanks to the warm welcome I received at GMV, where I was allowed to study scientific articles from the beginning and was supported in my training. In particular, I want to thank my professional tutor Juan Miguel and my colleague Daniel Hurtado for the help they have provided me. I also want to thank my academic tutor Carlos Gregorio for his guidance during the course of this work. And especially, I want to thank my boyfriend Miguel, who has put up with me while balancing the master’s studies, the work at GMV, and the writing of this document.
}

\normalfont
\tableofcontents

\chapter{Introduction}
\label{ch:Introduction}

\section{Deep Learning Models}

\subsection{Multilayer Perceptrons}
A Multilayer Perceptron (MLP), also known as feedforward neural network, is a mathematical model mainly used in supervised learning. Essentially, an MLP is a directed acyclic graph which represents the composition of functions. Each function or \textbf{layer} is a collection of neurons. A neuron is defined as:

\begin{equation}
    \label{eqn:neuron}
    y = f(\mathbf{x}; \theta) = \omega^T \mathbf{x} + b, \qquad \theta = (\omega, b)
\end{equation}

The term \textit{perceptron} refers to a linear classifier consisting of one layer \cite{rosenblatt1958}, $f(\mathbf{x}; \theta) = \mathbb{I}(\omega^T \mathbf{x} + b > 0)$.
Here, $\mathbb{I}(a>0)$ is the Heaviside function, which is non-differentiable. In MLP, the activation function from the original perceptron $\mathbb{I}$ is usually replaced by another differentiable function $\psi \colon \mathbb{R} \to \mathbb{R}$. The internal layers of MLP are usually named \textbf{hidden layers}, the last one is called \textbf{output layer}. The dimensionality of the hidden layers determines the \textbf{width} of the neural network \cite{goodfellow2016}. Each layer $l$ consists of many units $\mathbf{z}_l$ which are computed as a linear transformation of the units from the previous layer $l-1$ passed element-wise through the activation function \cite{murphy2022}:
\begin{equation*}
    \mathbf{z}_l = \psi_l (\mathbf{W}_l \mathbf{z}_{l-1} + \mathbf{b}_l)
\end{equation*}

We can choose different activation functions that will define our model and impact in the performance of the training. If we use a linear activation function $\psi_l (x) = K_l x$ then our neural network becomes just a linear model \cite{murphy2022}, that's why usually non-linear activation functions are used, since we would like to capture non-linear relationships between the input data and the output. Also, the Universal Approximation Theorem states that a neural network with a single hidden layer and non-linear activation functions can approximate any continuous function on a compact subset of $\mathbb{R}^n$ to any desired degree of accuracy.
Since the goal of an MLP is to approximate some function $f^*$ (for example, for a classifier $y = f^*(\mathbf{x})$), non-linear activation functions are necessary to achieve this level of approximation.
However, they can be used when we know the problem is linearly separable (think of two separate clusters of points in $\mathbb{R}^2$ in a classifier model) or in the output layer, for example when the MLP acts as a regression model.\\
If we change the Heaviside function by the sigmoid (logistic) function $\sigma(x) = \frac{1}{1+e^{-x}}$ we get a smooth approximation of $\mathbb{I}$. Another choice of activation function could be $tanh(x) = \frac{e^x - e^{-x}}{e^x + e^{-x}}$ which has a similar shape but its image is $(-1,1)$. Both are valid activation function and have been used. However, since we need to compute gradients in the optimization process in order to update the weights, a $\mathbf{0}$ gradient would be a problem, since it will make hard to train a model using gradient descent (vanishing gradient problem).
Both functions have an almost horizontal slope (gradient near 0) for large positive and negative inputs. Also, $\lVert \frac{d}{dx} \sigma(x) \rVert = \lVert \sigma(x)(1-\sigma(x)) \rVert \leq \frac{1}{4} < 1$, therefore several multiplications will quickly approximate to zero. Although they are used in practice (for example, $\sigma(\cdot)$ is used in the output layer for binary regression problems), in order to train deep models we need non-saturating activation functions for the hidden layers.
One of the most used is \textit{rectified linear unit}: $ReLU(x) = \max (a,0) = a \mathbb{I}(a>0)$. The gradient of $ReLU'(z) \neq 0$ as long as $z$ is positive: this function don't require input normalization to prevent them from saturating.
As stated in \cite*{glorot2011}, the rectifier activation function allows a network to easily obtain sparse representations, the \textit{hard} saturation (gradients being exactly 0) help supervised training: experimental results suggest that networks with ReLU activation functions in the hidden layers have better convergence performance than using sigmoid \cite{krizhevsky2017}.
There has been a lot of activation functions proposed in the last years, [TERMINAR ESTO] .\\

The neural network tries to approximate the target function $\mathbf{y} = F(\mathbf{x})$, the \textit{true} relation between the variables. The network, $f(x;\theta)$, \textit{learns} by searching the parameters $\boldsymbol{\theta}$ that minimizes a loss function $J(\boldsymbol{\theta})$, which measures the distance between the output and the target or the proximity between probability densities of random variables \cite*{calin2020}. If we consider a $k$-class classification problem with $\mathcal{X} \subset \mathbb{R}^d$ the feature space and $\mathcal{Y} = \{1,\dots, k\}$ the label space, given the dataset $D = \{(\mathbf{x}_1, y_1),\cdots,(\mathbf{x}_n, y_n)\}$ and a MLP $f\colon \mathcal{X} \to \mathbb{R}^k$ with a softmax as the output layer.
For any loss function $J(\boldsymbol{\theta}) = J(f(\mathbf{x}, \boldsymbol{\theta}), y)$, assuming there is a joint distribution $P(\mathbf{x},y)$ over $\mathcal{X}$ and $\mathcal{Y}$, the \textbf{risk} of $f$ is defined as $R_J (f) = \mathbb{E}\Bigl[ J(f(\mathbf{x}; \boldsymbol{\theta}), y) \Bigr] = \int J(f(\mathbf{x}; \theta), y) dP(\mathbf{x}, y)$ and the \textbf{empirical risk} is defined as $\hat{R}_J (f) = \mathbb{E}_D \Bigl[ J(f(\mathbf{x}; \boldsymbol{\theta}), y) \Bigr]$ assuming that $\{(\mathbf{x}_i, y_i)\}_{i=1}^n$ are IID samples from $P(\mathbf{x}, y)$.
Since the nonlinearity of an MLP makes the loss function to become non-convex, it's trained using iterative, gradient-based optimizers. Most neural networks, and in particular the models that will be used in this work, are trained using maximum likelihood: the loss function is the negative log-likelihood \cite{goodfellow2016}. That is, we will compute the \textit{cross-entropy} between the training data and the model distribution:

\begin{equation}
    J(\theta) = -\mathbb{E}_{\mathbf{x}, \mathbf{y} \sim \hat{\rho}_{data}} \log \rho_{model} (\mathbf{y} | \mathbf{x})
\end{equation}
In a $k$-class classification model with a cross-entropy loss function, the empirical risk to minimize is $\hat{R}_J (f) = -\frac{1}{n}\sum_{i=1}^n \sum_{j=1}^k \mathbf{y}_{ij} \log f_j (\mathbf{x}_i; \boldsymbol{\theta})$, where $\mathbf{y}_{ij}$ is the j-th element of the one-hot encoded label of $\mathbf{x}_i$ such that $\mathbf{1}^T \mathbf{y}_i = 1$, $\forall i$, and $f_j$ is the j-th element of the softmax output layer of $f$ \cite{zhang2018}.       \\


\subsection{Convolutional Neural Networks}


\section{Other ML models}

\chapter{Federated Learning}
\label{ch:Federated_Learning}

\section{Introduction}
In the rapidly evolving landscape of artificial intelligence and machine learning, Federated Learning (FL)
has emerged as a paradigm that addresses key challenges related to privacy, data security,
 and decentralized computing. Federated Learning represents a novel approach to model training, allowing
 machine learning models to be trained collaboratively across multiple decentralized devices or servers
 without exchanging raw data \cite{mcmahan2023a}.

 Unlike traditional centralized approaches, where data is collected and processed in a central server, FL enables training on local devices following a scheme of decentralized model training.
 This decentralization ensures that data remains on the device, granting a certain degree of privacy.
 In a centralized setting, the trained model is updated based on the complete dataset, which is stored in a unique server. In the federated setting, data is distributed across local devices (parties) and training happens locally. This can be problematic, since the source
 of the data differ, the data can differ in various ways: unbalanced datasets, different distributions, etc. This is one of the main challenges of FL:
 \textbf{non-IID data}, since this will negatively affect the performance of the model \cite{li2020}, \cite{zhao2018}, \cite{li2021}.

Horizontal Federated Learning (HFL) and Vertical Federated learning (VFL) are two variations of the federated learning paradigm that differ in how they distribute and collaborate on data.

\begin{itemize}
    \item \textbf{HFL:} Each party has a portion of the overall dataset, each party holds a different subset of examples but for the same features.
    \item \textbf{VFL:} The data  is vertically partitioned, each party has different features for the same set of examples (rows).
\end{itemize}

HFL and VFL are not mutually exclusive, in some cases a combination of both schemes may be applied. This work will focus on HFL. Also, we will only study Cross-Silo Federated Learning (Cross-Silo FL),
which is a variation of FL that addresses the scenario where data is distributed across different organizations, usually few parties, each maintaining control over its own data.
This setting is particularly relevant in industries where different organizations need to collaborate on machine learning task, such as healthcare (hospitals collaborating on medical research), finance (banks collaborating on fraud detection), epidemiological studies (international public health agencies studying disease spread),
smart cities (urban planning authorities collaborating on public services optimization), etc. Ensuring interoperability between different silos is a huge challenge, since there needs to be a fixed standard in data format, structures and processing capabilities accross different organizations.

We will begin by studying the FedAvg algorithm \cite{mcmahan2023a}, which is de facto approach for Federated Learning (FL). We will establish notation, examine some of its properties, and explore issues that arise when data is not independently identically distributed (statistical heterogeneity). Following that, various approaches that have been proposed to address this problem will be developed, and the performance of each will be analyzed across different training architectures.

\section{FedAvg}

Let $D = \{(\mathbf{x}, y)\}$ be the global dataset\footnote{Here, the global dataset is the union of the different local datasets, $D = \cup_{i=1}^N D^i$ . In practical cases, there is no such dataset in order to ensure data privacy. However, we will consider it to conduct a performance study of the various algorithms.} and $D^i \subset D$ the $i$-th party's local dataset, $i=1,...,N$.
Let $\omega_g^t$ and $\omega_i^t$ be the global model and the local model of the $i$-th party in round $t\in \{1,...,T\}$, respectively.

\chapter{Multi-Party Computation}

\section{Defining Multi-Party Computation}

\section{Protocol Examples}

\section{Private Set Intersection}

\include{chapter4}
\backmatter
\chapter{Appendix A}

\begin{figure}[H]
    \centering

    \begin{subfigure}{\linewidth}
        \centering
        \includegraphics[width=0.8\linewidth]{figures/2-Federated_Learning/FedProx_Dirichlet_1_mu_0.001.png}
    \end{subfigure}
    \vspace{1em} % Space between images

    \begin{subfigure}{\linewidth}
        \centering
        \includegraphics[width=0.8\linewidth]{figures/2-Federated_Learning/FedProx_Dirichlet_1_mu_0.01.png}
    \end{subfigure}
    \vspace{1em} % Space between images

    \begin{subfigure}{\linewidth}
        \centering
        \includegraphics[width=0.8\linewidth]{figures/2-Federated_Learning/FedProx_Dirichlet_1_mu_0.1.png}
    \end{subfigure}
    \vspace{1em} % Space between images

    \begin{subfigure}{\linewidth}
        \centering
        \includegraphics[width=0.8\linewidth]{figures/2-Federated_Learning/FedProx_Dirichlet_1_mu_1.png}
    \end{subfigure}

    \caption{Local metrics for 3 clients in 50 communication rounds using FedProx with a Non-IID setting over the CIFAR10 dataset, $\mu \in \{0.001, 0.01, 0.1, 1\}$. Label distribution skew using the Dirichlet distribution with $\boldsymbol{\beta} = (1,1,1)$ }
    \label{fig:FedProx_Non_IID_Dirichlet_1}
\end{figure}


\begin{figure}[H]
    \centering

    \begin{subfigure}{\linewidth}
        \centering
        \includegraphics[width=0.8\linewidth]{figures/2-Federated_Learning/FedProx_LabelsPerParty_mu_0.001.png}
    \end{subfigure}
    \vspace{1em} % Space between images

    \begin{subfigure}{\linewidth}
        \centering
        \includegraphics[width=0.8\linewidth]{figures/2-Federated_Learning/FedProx_LabelsPerParty_mu_0.01.png}
    \end{subfigure}
    \vspace{1em} % Space between images

    \begin{subfigure}{\linewidth}
        \centering
        \includegraphics[width=0.8\linewidth]{figures/2-Federated_Learning/FedProx_LabelsPerParty_mu_0.1.png}
    \end{subfigure}
    \vspace{1em} % Space between images

    \begin{subfigure}{\linewidth}
        \centering
        \includegraphics[width=0.8\linewidth]{figures/2-Federated_Learning/FedProx_LabelsPerParty_mu_1.png}
    \end{subfigure}

    \caption{Local metrics for 3 clients in 50 communication rounds using FedProx with a Non-IID setting over the CIFAR10 dataset, $\mu \in \{0.001, 0.01, 0.1, 1\}$.  Label distribution skew, the first client data from 2 classes, the second client
from 3 classes and the third client from the 5 remaining classes.}
    \label{fig:FedProx_Non_IID_LabelsPerParty}
\end{figure}




\begin{figure}[H]
    \centering

    \begin{subfigure}{\linewidth}
        \centering
        \includegraphics[width=0.8\linewidth]{figures/2-Federated_Learning/FedProx_QuantitySkew_Dir_05_Mu_0.001.png}
    \end{subfigure}
    \vspace{1em} % Space between images

    \begin{subfigure}{\linewidth}
        \centering
        \includegraphics[width=0.8\linewidth]{figures/2-Federated_Learning/FedProx_QuantitySkew_Dir_05_Mu_0.01.png}
    \end{subfigure}
    \vspace{1em} % Space between images

    \begin{subfigure}{\linewidth}
        \centering
        \includegraphics[width=0.8\linewidth]{figures/2-Federated_Learning/FedProx_QuantitySkew_Dir_05_Mu_0.1.png}
    \end{subfigure}
    \vspace{1em} % Space between images

    \begin{subfigure}{\linewidth}
        \centering
        \includegraphics[width=0.8\linewidth]{figures/2-Federated_Learning/FedProx_QuantitySkew_Dir_05_Mu_1.png}
    \end{subfigure}

    \caption{Local metrics for 3 clients in 50 communication rounds using FedProx with a Non-IID setting over the CIFAR10 dataset, $\mu \in \{0.001, 0.01, 0.1, 1\}$.  Label quantity distribution skew using $Dir(\boldsymbol{\beta})$, $\boldsymbol{\beta} = (0.5,0.5,0.5)$  }
    \label{fig:FedProx_Non_IID_LabelQuantitySkeqDir_05}
\end{figure}




\begin{figure}[H]
    \centering

    \begin{subfigure}{\linewidth}
        \centering
        \includegraphics[width=0.8\linewidth]{figures/2-Federated_Learning/FedProx_NoiseLevel_Mu_0.001.png}
    \end{subfigure}
    \vspace{1em} % Space between images

    \begin{subfigure}{\linewidth}
        \centering
        \includegraphics[width=0.8\linewidth]{figures/2-Federated_Learning/FedProx_NoiseLevel_Mu_0.01.png}
    \end{subfigure}
    \vspace{1em} % Space between images

    \begin{subfigure}{\linewidth}
        \centering
        \includegraphics[width=0.8\linewidth]{figures/2-Federated_Learning/FedProx_NoiseLevel_Mu_0.1.png}
    \end{subfigure}
    \vspace{1em} % Space between images

    \begin{subfigure}{\linewidth}
        \centering
        \includegraphics[width=0.8\linewidth]{figures/2-Federated_Learning/FedProx_NoiseLevel_Mu_1.png}
    \end{subfigure}

    \caption{Local metrics for 3 clients in 50 communication rounds using FedProx with a Non-IID setting over the CIFAR10 dataset, $\mu \in \{0.001, 0.01, 0.1, 1\}$. Feature distribution skew using with noise level $\sigma = 0.5$}
    \label{fig:FedProx_Non_IID_NoiseLevel_05}
\end{figure}


\printbibliography
\end{document}
